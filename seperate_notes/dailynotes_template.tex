%Declaring the document type. This can be article, report, book, letter etc.
\documentclass{article}

%The preamble: Place for declaration of recquired packages, defining styles, colors, title etc.
%---------------------------------------------------------------------------------------------%

%Declaring Packages%
\usepackage{amsmath} %equation styles
\usepackage{amsfonts} %fonts
\usepackage{amssymb} %math symbols
\usepackage{amsthm} %custom document divisions like, Definition, Example, proof, solution, problems etc.,
\usepackage{graphicx} %to include images
\usepackage{verbatim} % for code display and block comment using \begin{comment}...\end{comment} environment.
\usepackage{lipsum} %package for generating dummy text
% we can include more packages as required like, tikz,bibliography packages, etc.


%-------------------------------------------------------------------------------------------------%

%defining new theorem styles: Example, Problem, Solution, Proof using amsthm


\newtheorem{example}{Example}[section]
\newtheorem{problem}{Problem}[section]
\newtheorem{lem}{Lemma}[section]
\newtheorem{thm}{Theorem}[section]
\newtheorem{define}{Definition}[section]

%----------------------------------------------------------------------------------------------%
\title{Topic of the day!}
\author{The author.}
\date{dd/mm/yy}
%------------------------------------------------------------------------------------------------%


\begin{document}
\maketitle %printing the title, author name, date

% Article has the Document divisions going from \section{title} -> \subsubsection{title} -> \subsection{title}
% similarly the content can be divide into \paragraph{title} -> \subparagraph{title}
\section{First Section}

\subsection{Lists, Matrix and Table}
\subsubsection{Lists}
\begin{itemize}
	\item Lists are within \verb|begin{itemize}...\end{itemize}|
	\item Lists are listed using \verb*|\item| it displays a bullet point.
	\item Other ways to have lists are using \verb*|\begin{enumerate}...\end{enumerate}| environment. Which again uses \verb*|\item| to list content.
\end{itemize}
\begin{verbatim}
	\bein{itemize}
		\item line 1
		\item line 2
		\item line 3
	\end{itemize}
\end{verbatim}

\subsubsection{Matrices}
\begin{enumerate}
	\item Matrices are displayed within math mode (\verb*|$...$|, \verb*|$$...$$|) or,
	\item \verb*|\begin{Bmatrix}...\end{Bmatrix}|,\verb*|\begin{Pmatrix}...\end{pmatrix}|, \verb*|\begin{Vmatrix}...\end{Vmatrix}|
	\item Each element of the matrix is sepearated using \& and each row upto the last one is ended using the linebreak(nextline command)\verb*|\\|
	\item Bmatrix displays matrix within \{Curlybraces\}, Pmatrix displays it within (paranthesis), Vmatrix displays it as a Determinant.
\end{enumerate}
\begin{verbatim}
	\begin{equation}
	\begin{pmatrix}
		1 & 2 & 3 \\
		4 & 5 & 6\\
		7 & 8 & 9\\
	\end{pmatrix}
	\end{equation}
\end{verbatim}
\begin{equation}
	\begin{pmatrix}
		1 & 2 & 3 \\
		4 & 5 & 6\\
		7 & 8 & 9\\
	\end{pmatrix}
\end{equation}

\subsubsection{Tables}
\par Displaying tables is similar to matrices but with \verb*|\begin{tabular}{}...\end{tabular}| environment. The extra \{$|c|c|c|$\} is to specify the number of coloumns. It is delimited by Vertical bars $|$ to denote borders/seperations.
\begin{verbatim}
	\begin{center} %to center the table
	\begin{tabular}{|c|c|c|}		
		\hline 
		Header & Header & Header \\
		\hline
		item & Item & Item \\ 
		item & Item & Item \\
		\hline
	\end{tabular}
	\end{center}
\end{verbatim}
\begin{center}
\begin{tabular}{|c|c|c|} %the extra option 
	%Table head
	\hline %for outline - horizontal line
	Header & Header & Header \\
	\hline
	item & Item & Item \\ 
	item & Item & Item \\
	\hline
\end{tabular}
\end{center}



\subsection{Theorem, Lemma, Problem, Definition, proof and solution}

%using our own environments made using amsthm
%--------------------------------------------%
All of these theorem, lemma etc., environments are defined in the preabmble as follows,
\begin{verbatim}
	\newtheorem{example}{Example}[section]
	\newtheorem{problem}{Problem}[section]
	\newtheorem{lem}{Lemma}[section]
	\newtheorem{thm}{Theorem}[section]
	\newtheorem{define}{Definition}[section]
\end{verbatim}
\rule{\textheight}{0.4pt}
%theorem environment
\begin{thm} 
	content...
	\begin{proof}[\textbf{proof}]
		content...
	\end{proof}
\end{thm}

\begin{verbatim}
	\begin{thm} 
		content...
		\begin{proof}[\textbf{proof}]
			content...
		\end{proof}
	\end{thm}
\end{verbatim}

\rule{\textheight}{0.4pt}

%lemma environment
\pagebreak
\begin{lem}
	content...
	\begin{proof}[\textbf{proof}]
		content...
	\end{proof}
\end{lem}

\begin{verbatim}
	\begin{lem}
		content...
		\begin{proof}[\textbf{proof}]
			content...
		\end{proof}
	\end{lem}
\end{verbatim}
\rule{\textheight}{0.4pt}

%problem environment

\begin{problem}
	A dummy problem statement can be produced inside this environment. With inline equations or displaymode equations.
	\begin{proof}[\textbf{solution}]%the begin proof environment is improvisable with the [option] for solutions, proofs,etc
		here we can have our solutions inside the begin{proof}...end{proof} environment,\\
		We can have equations and other environments as needed.
	\end{proof}
\end{problem}

\begin{verbatim}
	\begin{problem}
		A dummy problem statement can be produced inside this environment. With inline equations or displaymode equations.
		\begin{proof}[\textbf{solution}]
			here we can have our solutions inside the begin{proof}...end{proof} environment,\\
			We can have equations and othen environments as needed.
		\end{proof}
	\end{problem}
\end{verbatim}

\rule{\textheight}{0.4pt}


%definition environment

\begin{define}
	content...
\end{define}

\begin{verbatim}
	\begin{define}
		content...
	\end{define}
\end{verbatim}

\rule{\textheight}{0.4pt}

\end{document}
