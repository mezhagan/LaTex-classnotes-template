\documentclass{article}

\usepackage{amsmath} 
\usepackage{amsfonts} 
\usepackage{amssymb} 
\usepackage{amsthm} 
\usepackage{graphicx}
\usepackage{verbatim}
\usepackage{lipsum} 

\newtheorem{example}{Example}[section]
\newtheorem{problem}{Problem}[subsection]
\newtheorem{lem}{Lemma}[section]
\newtheorem{thm}{Theorem}[section]
\newtheorem{define}{Definition}[section]

\title{A quick review of special relativity}
\author{Gurunandha Elamezhagan S}
\date{\ }

\begin{document}
\maketitle
\section{The class lecture}
The derivations and discussion made in class, seperated into appropriate sections.
\begin{itemize}
	\item use appropriate environments.
	\item utilize itemize, equation, align/gather, definition environments.
	\item you can create your own divisions as you like.
	\item Make sure the document is structured.
	\item Use the cheat sheets/google/chatgpt if you feel stuck. If you paste the code block or error into chatgpt it will provide you approximate solutions.
\end{itemize}
\subsection{Derivation of ---------}
some discussion, and starting the derivation 
\begin{gather}
	equation 1\\
	equation 2 ==xxx
\end{gather}
\subsection{Discussion of ------}
\lipsum[1-2]


\section{Problems}
\subsection{Class Problems}
\begin{problem}
	The problem statement.
	\begin{proof}[Solution]
		Type the solution of the problem here. The problem will be automatically numbered.
	\end{proof}
\end{problem}
\begin{problem}
	The problem statement.
	\begin{proof}[Solution]
		Type the solution of the problem here. The problem will be automatically numbered.
	\end{proof}
\end{problem}
\subsection{Homework Problems}
\begin{problem}
	The problem statement.
	\begin{proof}[Solution]
		Type the solution of the problem here. The problem will be automatically numbered.
	\end{proof}
\end{problem}
\end{document}